\documentclass{beamer}
\title{Day 2: Conditionals, Loops, Basic Functions}
\author{Andrew Luo}
\usepackage{listings}
\usepackage{color}
\usepackage{graphicx}
\usepackage{upquote}

\usepackage{hyperref}
\hypersetup{
	colorlinks = true,
	linkcolor=white,
	urlcolor=blue,
}



\usetheme{Berkeley}
% adding javascript language capabilities for code snippets
\definecolor{lightgray}{rgb}{.9,.9,.9}
\definecolor{darkgray}{rgb}{.4,.4,.4}
\definecolor{purple}{rgb}{0.65, 0.12, 0.82}
\lstdefinelanguage{JavaScript}{
  keywords={let, break, case, catch, continue, debugger, default, delete, do, else, false, finally, for, function, if, in, instanceof, new, null, return, switch, this, throw, true, try, typeof, var, void, while, with},
  morecomment=[l]{//},
  morecomment=[s]{/*}{*/},
  morestring=[b]',
  morestring=[b]",
  ndkeywords={class, export, boolean, throw, implements, import, this},
  keywordstyle=\color{blue}\bfseries,
  ndkeywordstyle=\color{darkgray}\bfseries,
  identifierstyle=\color{black},
  commentstyle=\color{darkgray}\ttfamily,
  stringstyle=\color{red}\ttfamily,
  sensitive=true
}

\lstset{
   language=JavaScript,
   extendedchars=true,
   basicstyle=\footnotesize\ttfamily,
   showstringspaces=false,
   showspaces=false,
   numbersep=9pt,
   tabsize=2,
   breaklines=true,
   showtabs=false,
   captionpos=b
}

\begin{document}
\maketitle
\begin{frame}
\frametitle{Variable Names}
\begin{itemize}
	\item names cannot start with a digit
		\begin{itemize}
			\item ex. 95Andrew is illegal variable name
		\end{itemize}
	\item may include dollar signs (\$) and underscores (\_) but no other special characters
	\item cannot include keywords
		\begin{itemize}
			\item ex. let, var, const
		\end{itemize}
	\item usually use camelCase by convention
		\begin{itemize}
			\item first letter is lowercase and all others are uppercase
			\item ex. andrewLuoIsCool
		\end{itemize}
\end{itemize}
\begin{figure}[htbp]
	\centerline{\includegraphics[scale=0.08]{camelCase.png}}
	\caption{camelCase makes variable names look like the shape of a camel's back}
	\label{Fig 1}
\end{figure}
\end{frame}

\begin{frame}[fragile]
\frametitle{String Escape Characters}
\begin{itemize}
	\item create a line break with \textbackslash n 
	\item \begin{lstlisting}
	console.log("Hello \n World");
	// outputs -> Hello
	//            World
	\end{lstlisting}
	\item create a tab with \textbackslash t
\end{itemize}
\end{frame}

\begin{frame}[fragile]
\frametitle{Template Literal}
\begin{itemize}
	\item use \`{} \`{} backticks, then you can insert expressions into strings using \$\{\} 
	\item ex. \begin{lstlisting}
	let myName = "Bob";
	console.log(`Hello ${myName}`);
	// will output -> Hello Bob
	\end{lstlisting}
\end{itemize}
\end{frame}

\begin{frame}
\frametitle{Comparison Operators}
\begin{itemize}
	\item an \emph{expression} is a statement that takes in multiple values and produces a new value
	\item comparison expressions produce either true or false
	\item \(>\)  
	\item \(<\)
	\item \(>\)= 
	\item \(<\)=
	\item ==
		\begin{itemize}
			\item only checks if values are the same 
			\item ex. 5 == "5" will return true
		\end{itemize}
	\item ===
		\begin{itemize}
			\item checks if both value and types are the same
			\item ex. 5 === "5" will return false
		\end{itemize}
\end{itemize}
\end{frame}

\begin{frame}
\frametitle{Comparing Strings}
\begin{itemize}
	\item strings can be compared
	\item the order is roughly alphabetic
	\item lowercase always comes before uppercase
	\item compares each letter one by one
	\item ex. happy comes before Happy, but after hapoy
\end{itemize}
\end{frame}

\begin{frame}[fragile]
\frametitle{Type Coercion}
\begin{itemize}
	\item when comparing two things of different type, Javascript converts them into the same type
	\item NaN
		\begin{itemize}
			\item Not a Number
		\end{itemize}
	\item you can also test if a variable actually holds a value
	\item ex. \begin{lstlisting}
	let myVariable;
	console.log(myVariable == undefined);
	// returns true because myVariable has not been initialized
	\end{lstlisting}
\end{itemize}
\end{frame}

\begin{frame}
\frametitle{Logical Operators}
\begin{itemize}
	\item \&\&
		\begin{itemize}
			\item "and"
			\item evaluates to true if both operands are true
		\end{itemize}
	\item \textbar \textbar
		\begin{itemize}
			\item "or"
			\item evaluates to true if one of the operands is true
		\end{itemize}
	\item !
		\begin{itemize}
			\item "not"
			\item evaluates to true if the operand is false
		\end{itemize}
\end{itemize}
\end{frame}

\begin{frame}
\frametitle{Short Circuit Evaluation}
\begin{itemize}
	\item for \textbar \textbar ,if the first operand is true, then the second operand will not be evaluated
	\item for \$\$ ,if the first operand is false, then the second operand will not be evaluated
	\item use this behaviour to create a fallback to default value
		\begin{itemize}
			\item "console.log(null \textbar "Hello")"
			\item will fall back to "Hello"
			\item 0, NaN,undefined,null and empty string "" will evaluate to false
		\end{itemize}
\end{itemize}
\end{frame}

\begin{frame}[fragile]
\frametitle{Unary, Binary, Ternary Operators}
\begin{itemize}
	\item an \emph{operand} is a variable which is taken by an operation
	\item unary operators only have 1 operand
		\begin{itemize}
			\item ex. typeof operator
			\item \begin{lstlisting}
			let myVarible = "Andrew";
			console.log(typeof myVariable);
			// will output -> string
				\end{lstlisting}
		\end{itemize}
	\item ternary operators have 3 operands
		\begin{itemize}
			\item \begin{lstlisting}
			condition ? if true : if false	
			\end{lstlisting}
		\end{itemize}

\end{itemize}
\end{frame}

\begin{frame}[fragile]
\frametitle{Conditional Execution}
\begin{itemize}
	\item if, else statements
	\item form: \begin{lstlisting}
	if (expression) {
	//executes if true
	}
	else {
	//executes if false
	}
	\end{lstlisting}
	\item if, else if statements
	\item form: \begin{lstlisting} 
	if (expression1) {
	//executes if expression is true
	}
	else if (expression2) {
	//executes if expression 1 is false and expression 2 is true
	}
	else {execuetes if 1 and 2 are false} 
	\end{lstlisting}
\end{itemize}
\end{frame}

\begin{frame}[fragile]
\frametitle{While Loop}
\begin{itemize}
	\item continues executing until expression is false
	\item you need to put an updating statement in the body that eventually makes the expression false
	\item failure to do so leads to an \emph{infinite loop}
	\item ex. \begin{lstlisting}
		let i = 0;
		while (i < 10)
		{
			console.log("Hello World");
			i = i + 1;
		}
		// how many times does this output hello world?
		\end{lstlisting}
\end{itemize}
\end{frame}

\begin{frame}[fragile]
\frametitle{Do While Loop}
\begin{itemize}
	\item same as while loop but the loop executes at least once, updating statement happens at the end of the loop body
	\item ex. \begin{lstlisting}
	let i = 0;
	do {
		console.log("Hello");
		i = i + 1;
	}
	while (i < 0);
	// still prints Hello once
	\end{lstlisting}
\item notice how you want to indent your code for easy readability
\end{itemize}
\end{frame}

\begin{frame}[fragile]
\frametitle{For Loop}
\begin{itemize}
	\item useful when you know how many times you want to iterate (repeat)
	\item \begin{lstlisting}
	for (let i = 0; i < 10; i++){
		console.log("Hello World");
	}
	// prints "Hello World" 9 times
	\end{lstlisting}
\end{itemize}
\end{frame}


\begin{frame}
\frametitle{Incrementing and Decrementing}
\begin{itemize}
	\item i = i + 1 is known as \emph{incrementing} the variable i
		\begin{itemize}
			\item incrementing is increasing the numeric value by 1
			\item decrementing is decreasing the numeric value by 1
		\end{itemize}
	\item there are shortcuts for incrementing and decrementing
	\item i += 1
	\item i++
	\item i -= 1
	\item i - -
\end{itemize}
\end{frame}

\begin{frame}[fragile]
\frametitle{Functions}
\begin{itemize}
	\item the purpose of functions are to keep your code DRY ("Don't Repeat Yourself")
	\item code snippets that are reusable through code
	\item there are two ways of making functions
	\begin{itemize}
		\item Old Way
		\item \begin{lstlisting}
		function myFunction() {
		// function body
		}
		\end{lstlisting}
	\item New Way: ES6 Functions
	\item \begin{lstlisting}
	let myFunction = () => {
	//function body
	}
	\end{lstlisting}
	\end{itemize}
\end{itemize}
\end{frame}

\begin{frame}[fragile]
	\frametitle{Parameters and Arguments}
	\begin{itemize}
		\item you can pass values to a function using\emph{arguments} when you invoke the function
		\item the function receives the arguments as \emph{parameters}
		\item you need to specify what the function receives between the parentheses of the declaration
		\item ex. \begin{lstlisting}
		function hello(name) {
			console.log(`Hello ${name}`);
		}

		//later I invoke the function using ()
		hello("Andrew");
		// prints "Hello Andrew"
		\end{lstlisting}
	\end{itemize}
\end{frame}

\begin{frame}[fragile]
\frametitle{Return Value}
	\begin{itemize}
		\item when the function is invoked, it can return a value
		\item ex. \begin{lstlisting}
		function hello(name) {
			return `Hello ${name}`;
		}
		//later I invoke the function, but I then can store the return value into another variable

		let returnValue = hello("Andrew");
		console.log(returnValue);
		// prints "Hello Andrew"
		\end{lstlisting}
	\end{itemize}
\end{frame}

\end{document}
