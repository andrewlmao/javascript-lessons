\documentclass{beamer}
\title{Day 1: Introduction}
\author{Andrew Luo}
\usepackage{listings}
\usepackage{color}

\usepackage{hyperref}
\hypersetup{
	colorlinks = true,
	linkcolor=white,
	urlcolor=blue,
}



\usetheme{Berkeley}
% adding javascript language capabilities for code snippets
\definecolor{lightgray}{rgb}{.9,.9,.9}
\definecolor{darkgray}{rgb}{.4,.4,.4}
\definecolor{purple}{rgb}{0.65, 0.12, 0.82}
\lstdefinelanguage{JavaScript}{
  keywords={let, break, case, catch, continue, debugger, default, delete, do, else, false, finally, for, function, if, in, instanceof, new, null, return, switch, this, throw, true, try, typeof, var, void, while, with},
  morecomment=[l]{//},
  morecomment=[s]{/*}{*/},
  morestring=[b]',
  morestring=[b]",
  ndkeywords={class, export, boolean, throw, implements, import, this},
  keywordstyle=\color{blue}\bfseries,
  ndkeywordstyle=\color{darkgray}\bfseries,
  identifierstyle=\color{black},
  commentstyle=\color{darkgray}\ttfamily,
  stringstyle=\color{red}\ttfamily,
  sensitive=true
}

\lstset{
   language=JavaScript,
   extendedchars=true,
   basicstyle=\footnotesize\ttfamily,
   showstringspaces=false,
   showspaces=false,
   numbersep=9pt,
   tabsize=2,
   breaklines=true,
   showtabs=false,
   captionpos=b
}

\begin{document}
\maketitle
\begin{frame}
\frametitle{Contact}
\begin{itemize}
\item{\href{mailto:andrew.luo25@gmail.com}{andrew.luo25@gmail.com}}
\item{Go to my GitHub Slides for lesson slides}
	\begin{itemize}
		\item{\url{https://github.com/andrewlmao/javascript-lessons}}
	\end{itemize}
\end{itemize}
\end{frame}

\begin{frame}
\frametitle{Materials}
\begin{itemize}
	\item{Notebook for taking notes}
	\begin{itemize}
		\item{We will be writing some code by hand to better keep it in memory}
	\end{itemize}
	\item{CodingRooms Account}
		\begin{enumerate}
			\item{Go to \url{https://www.codingrooms.com/}}
			\item{Click "Sign Up" in the top right corner}
		\end{enumerate}
\end{itemize}
\end{frame}

\begin{frame}
\frametitle{What is Programming?}
\begin{itemize}
\item{the act of constructing a program}
	\begin{itemize}
		\item a \emph{program} is a set of instructions telling a computer what to do
		\item {allows us to do things in seconds that would normally take forever by hand}
	\end{itemize}
\item{a \emph{programming language} is a  constructed language used to communicate with computers}
	\begin{itemize}
		\item easier for humans to read
		\item a \emph{low-level language} is closest to the commands and functions that processor understands
			\begin{itemize}
				\item ex. the lowest level language is machine code (binary)
			\end{itemize}
		\item a \emph{high-level language} abstracts from the details of the  computer
			\begin{itemize}
				\item {closer to human language and semantics}
				\item {not understood by the computer}
				\item ex. Java, C++, Python
			\end{itemize}
	\end{itemize}
\end{itemize}
\end{frame}

\begin{frame}
\frametitle{Compiling vs. Interpreter}
\begin{itemize}
	\item {the job of a \emph{compiler} is to translate a high level language into machine code that the computer understands}	
	\begin{itemize}
		\item the programmer creates \emph{source code} file in a high-level language
		\item the source code is then passed to a compiler, which produces machine code (known as a \emph{binary})
		\item{examples include Java, C++}
	\end{itemize}
	\item{the job of an \emph{intepreter} is to interpret each line of code into language that the computer understands}
	\begin{itemize}
		\item there is no source code
		\item the interpreter acts as a middle person
		\item slower than compiled languages
		\item {examples include Python and Javascript}
		\item browsers contain Javascript interpreters that can run Javascript
	\end{itemize}
\end{itemize}
\end{frame}

\begin{frame}
	\frametitle{What is Javascript?}
	\begin{itemize}
		\item {one of the three fundamental technologies of the web}
			\begin{itemize}
				\item{Javascript}
				\item{HTML (HyperText Markup Language)}
				\item{CSS (Cascading Style Sheets)}
			\end{itemize}
		\item{think of modern websites as a house}
			\begin{itemize}
				\item HTML gives structure to websites, like the frame of a house
				\item CSS gives style to websites, like the paint of a house
				\item Javascript makes websites dynamic (can change), like doors and windows that can open and close
			\end{itemize}
		\item{originally written for adding programs to Netscape Browser in 1995}
		\item has nothing to do with Java
		\item implementation of the ECMAScript standard to ensure compatibility between different browsers
	\end{itemize}
\end{frame}

\begin{frame}[fragile]
	\frametitle{Hello World}
\begin{itemize}
	\item Write the following into your editor
	\item \begin{lstlisting}
console.log("Hello World");
\end{lstlisting}

\end{itemize}
\begin{itemize}
	\item this prints "Hello World" into the console
\end{itemize}
\end{frame}

\begin{frame}
\frametitle{console.log()}
\begin{itemize}
	\item prints whatever is passed into the parentheses into standard output
	\item we will elaborate on functions later
\end{itemize}
\end{frame}

\begin{frame}[fragile]
\frametitle{Declaring and Initializing Variables}
\begin{itemize}
\item a \emph{variable} holds a particular value
\item each variable has a type and name
\item variables are \emph{dynamically type}
	\begin{itemize}
		\item Javascript automatically defines the type
		\item you only have to give a name
	\end{itemize}

\end{itemize}
\begin{lstlisting}
let myVariable = 1;
\end{lstlisting}
\begin{itemize}
	\item \emph{let, const, and var} are keywords that create a variable
	\item \emph{myVariable} is the variable name
	\item \emph{=} is the assignment operator
	\item the number 1 is being assigned to myVariable
	\item every statement in Javascript ends with semicolon ; or nothing
\end{itemize}
\end{frame}

\begin{frame}
\frametitle{Var, Const, Let}
\begin{itemize}
	\item \emph{var} is a global scope variable
	\item \emph{const} is a variable whose value cannot change after being initialized
	\item \emph{let} is a block scope variable
		\begin{itemize}
			\item blocks are denoted by curly braces \{\}
		\end{itemize}
\end{itemize}
\end{frame}

\begin{frame}
\frametitle{Data Types}
\begin{itemize}
	\item number
	\item string
		\begin{itemize}
			\item{denoted by quotation marks}
		\end{itemize}
	\item object
	\item symbol
	\item boolean
		\begin{itemize}
			\item {true or false}
		\end{itemize}
	\item null
		\begin{itemize}
			\item means no value 
		\end{itemize}
	\item undefined
		\begin{itemize}
			\item variable has been defined but not initalized
		\end{itemize}
\end{itemize}
\end{frame}

\begin{frame}
\frametitle{Math Operations}
\begin{itemize}
	\item +
	\item -
	\item *
		\begin{itemize}
			\item multiplication
		\end{itemize}
	\item /
		\begin{itemize}
			\item division
		\end{itemize}
	\item \%
		\begin{itemize}
			\item modulus
			\item means remainder
			\item ex. 6 \% 4 = 2
		\end{itemize}
\end{itemize}
\end{frame}

\begin{frame}[fragile]
\frametitle{String Concatenation}
\begin{itemize}
	\item you can use the + operator to concatenate two strings
	\item \begin{lstlisting} 
	let myString = "Andrew" + "Luo" 
		\end{lstlisting}
	\item you can even use the + operator to add strings with 
		\begin{itemize}
			\item objects
			\item numbers
			\item null
			\item undefined
		\end{itemize}
\end{itemize}
\end{frame}

\begin{frame}[fragile]
\frametitle{Comments}
\begin{itemize}
	\item one line comments use //
	\item multiline comments use /* */
\end{itemize}
\begin{lstlisting}
// This is a single line comment
/*
This is a 
multiline
comment
*/
\end{lstlisting}
\end{frame}

\begin{frame}
\frametitle{Sources}
\begin{itemize}
	\item{Eloquent Javascript 3rd Edition by Marijn Haverbeke}
\end{itemize}
\end{frame}
\end{document}
